\section*{Abstract}

For a long time, models of the Moho discontinuity have been created from a variety of different methods including gravitational and seismological studies. However, very few of these models developed have uncertainty estimates, this is especially the case where there is limited seismic data available. In areas such as South America and Africa due to the economic and environmental challenges, over vast regions of the continent, there are little to no seismic point estimates. For these areas to have relevant Moho models either gravitational data needs to be used or the seismic data has to be interpolated; there is no way to tell how accurate these methods are to attaining a true Moho depth model. To determine a method's accuracy a method of cross-validation specifically repeated random sub-sample validation will be used to quantify errors on a gravitationally derived model of South America with the help of seismic point estimates. The results from this cross-validation will evaluate the accuracy of gravitational models in regions where there is no seismic data to compare it to. Additionally, for regions where the model significantly underestimates the Moho depth in comparison to the seismic data, there is likely an unmodelled mass present. The Paraná Basin, South America is thought to have large igneous intrusions resulting in a shallower Moho than expected. In this study, these intrusions will be modelled in an attempt to decrease the errors on the Moho model currently used. The cross-validation gives an indication that the specific model used has a reasonably small misfit from the point estimates but that the size of the error may vary geographically across South America as the error values attained are an average for the whole model.



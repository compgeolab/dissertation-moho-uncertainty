\chapter{Introduction}

The Mohorovičić discontinuity or Moho for short is the physical boundary signified by the change in many properties such as mineralogy, density, and temperature among other things, but it is mainly known as the change from the crust to the upper mantle. Ever since the discovery of this boundary observed through a significant change in seismic p-wave velocity on either side of the Moho by seismologist Andrija Mohorovičić in 1909 there have been many people using multiple methods to try and quantify the depths to this boundary. Some estimates of the Moho discontinuity include \cite{Laske2013}, \cite{Assumpo2013}, and \cite{Reguzzoni2015}, all of which have slightly different models depending on types of methods and data. Geophysical techniques such as seismology and gravimetry have been utilized to estimate the depth of the discontinuity and topography over a local to a global scale. This issue is known as a geophysical inverse problem and uses data in the form of gravity, seismic, or another method to determine the depth to the Moho over a certain area and produce a model.
Why is this important, the determination of the Moho is critical for many reasons. One of the main problems that it helps solve is knowledge about how plates deform and move, it is not the answer to the whole issue, but it is a contributing factor. Especially in regions of continental crusts where their behaviour is somewhat less predictable and understood when compared to oceanic plates. In continental settings, the crustal thickness is indicative of the stress of the lithosphere and helps with the determination through models of inter-plate faulting that often lead to earthquake events. The Moho thickness can also provide an insight into geothermal heat flow helping with heat flux models of the Earth where often without these depths to the crust-mantle interface these values need to be estimated as are unknown. Finally, determination of the thickness of the crust improves understanding of where deeper Earth minerals may be found. Or where specific minerals have been found in the past correlating this to a specific Moho thickness may indicate as to the type of setting to find explore in the future.
In this paper, the moho model will be derived from gravitational data and removing the local gravity disturbances to get the regional field that is almost entirely based on the depth to the Crust-Mantle boundary with the help of other parameters that need to be constrained including the density, reference Moho, and regularisation parameter. The problem with using gravitational data to model the Moho is that gravity values calculated to be the regional field do not only include the effect of the discontinuity but also the effects of unmodelled masses in the Earth's crust that have not been removed when taking away all other effects that contribute to the strength of the raw gravity data. These unmodelled masses lead to the emergence of uncertainties in the model created, however with these uncertainties being of an unknown magnitude then it is difficult to quantify and correct for these unknown masses as the location, size, density, and number are impossible to determine. This is not just the case in the \cite{Uieda2016}, but almost all gravitationally derived Moho models suffer the same fate including \cite{vanderMeijde2013}, \cite{Tugume2013}, and \cite{Reguzzoni2015}. All of these models produced are generally accurate representations of the Moho surface however none of these includes errors or uncertainty values to coincide with their models. Although these error estimates are often not seen in seismologically derived Moho models either, one of the only recognizable papers that try to calculate the uncertainty of their estimates is \cite{Szwillus2019} who obtain uncertainties by interpolating Moho depth even then these uncertainties calculated are larger than the errors coming from the P-wave velocity.
In this paper, the aim is to find uncertainties in a gravitationally calculated Moho model by using cross-validation with seismic constraints. With the hopes of finding the difference between a model calculated from gravity data and seismic point estimates to see how good the gravitational estimates of the depth to the moho are where there are not any seismic points that the model can constrain to. Continents such as Europe and North America have extensive seismological surveys that span most of the area so it would be considered pointless to see how well this method of uncertainty estimation works as there are no substantial areas of land without seismic data. This is why South America has been chosen as most of the surveyed areas by either reflection or refraction are situated along the coast with few based towards the centre of the continent mainly due to the difficultly of surveying a result of the magnitude and density of vegetation in the Amazon rainforest. However, South America is also limited with seismic data due to lack of financial funding as surveys over a large scale may not be economically viable for countries or companies that are interested in carrying out one. So with the vast area of the Amazon having little to no seismic points to constrain the gravitationally computed Moho discontinuity model it will be a good test to see how good the model is when there are no seismic point estimates to compare to.
Therefore this paper will use the Moho model of South America created by \cite{Uieda2016} and implement a cross-validation approach to calculate the uncertainties through the average differences between gravitational estimates and seismic point estimates of the crustal thickness or depth to Moho. The method builds upon the cross-validation originally used in the paper by randomly selecting a training and testing set and calculating not only the uncertainty but how many seismic point constraints are needed to accurately quantify the size of this error. In addition to using cross-validation, in the synthetic models created of South America, the effect of trying to model a previously unknown mass will be tackled by adding in underplating in the Parana basin addressed in \cite{Mariani2013}. Through this the possibility of locating unmodelled masses could be determined, if the cross-validation of seismic point estimates provides a difference from the gravity model then it is likely that in the area there is a denser/less dense mass in the crust depending on if the difference is positive or negative.

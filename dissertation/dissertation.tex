% Dissertation template for the Geophysics programs of the University of
% Liverpool.
%
% This document sets the configuration and assembles the individual chapters
% (stored in separate .tex files). See the declaration.tex file for
% instructions on including scanned signature.

% Fill these in and they will be set throughout the document
\newcommand{\Name}{Aidan Hernaman}  % Note that you can use any Unicode character
\newcommand{\Degree}{Master of Science in the subject of Geophysics and Geology}
\newcommand{\Title}{Estimating the accuracy of Moho depth estimates from gravity inversion}

\documentclass[11pt,a4paper,oneside]{book}
% Full Unicode support for non-ASCII characters
\usepackage[utf8]{inputenc}
% Typographical rules for English
\usepackage[english]{babel}
% Handling figures in PNG, JPG, PDF, etc
\usepackage{graphicx}
% Better and more extensive maths
\usepackage{amsmath}
% Set the borders of the page
\usepackage[width=150mm,top=40mm,bottom=30mm,headsep=10mm,headheight=5mm]{geometry}
% Include links and metadata in PDFs
\usepackage[pdftex,colorlinks=true]{hyperref}
% Define command to insert month name and year as date
\usepackage{datetime}
% Nice styling for headers and footers
\usepackage{fancyhdr}
% To control the style of section titles
\usepackage{titlesec}
% Add the bibliography to the table of contents
\usepackage[nottoc,chapter]{tocbibind}
% Formatting the bibliography
\usepackage[round]{natbib}
% To insert dummy text into this template. Can be removed.
\usepackage{lipsum}

% Define a date format that is just the month and year
\newdateformat{monthyear}{\monthname[\THEMONTH], \THEYEAR}

% Customize how Chapter headings are displayed
\titleformat{\chapter}[display]{\normalfont\bfseries\centering}{\vspace{-2cm}\LARGE Chapter \thechapter}{5pt}{\Huge}

% Setup metadata for the PDF and link colour
\hypersetup{
    pdftitle={\Title},
    pdfauthor={\Name},
    pdfsubject={Dissertation submitted for the degree of \Degree{}},
    linkcolor=black,
    citecolor=black,
    filecolor=black,
    urlcolor=blue
}

% Set fancy headers
\usepackage{fancyhdr}
\pagestyle{fancy}
\fancyhf{}
\lhead{\fontsize{10pt}{0}\selectfont\itshape \nouppercase\leftmark}
\chead{}
\rhead{\fontsize{9pt}{0}\selectfont \thepage}
\cfoot{}
\renewcommand{\headrulewidth}{0pt}
%\renewcommand{\chaptermark}[1]{\markboth{#1}{}}

% Increase the line spacing
\renewcommand{\baselinestretch}{1.5}

\begin{document}

  % Gather all of the .tex files for individual sections into a single
  % document. To add text and edit, open the individual files added with the
  % "\include{}" command.

  \pagestyle{plain}

  % You don't need to edit the cover, it will be generated from the variables
  % set at the very top of this document.
  % Cover page for the dissertation. Uses variables defined in the main
% dissertation.tex document. No need to edit things here.

%\addtolength{\topmargin}{-3in}

\thispagestyle{empty}

\begin{figure}[t!]
  \begin{center}
    \includegraphics[width=0.7\textwidth]{figures/university-of-liverpool-logo}
  \end{center}
\end{figure}

\vspace*{3cm}

\begin{center}
  \textbf{\LARGE \Title{}}
  \\[10mm]
  {\Large by}
  \\[10mm]
  {\Large \Name}
  \\
  \vfill
  \begin{minipage}[t]{0.7\textwidth}
    A dissertation submitted to the
    Department of Earth, Ocean, and
    Ecological Sciences,
    University of Liverpool, in partial fulfilment of
    the requirements for the degree of
    \Degree{}.
  \end{minipage}
  \\[3cm]
  \monthyear\today
\end{center}


  \frontmatter

  % You need to edit the declaration to include an image of your signature. See
  % the declaration.tex file for instructions.
  % Official declaration page that is required. The date is automatically set
% when the PDF is generated. Place a PNG file of your signature in the
% "figures" folder called "signature.png" (this exact name!). DON'T ADD THIS TO
% THE GIT RESPOSITORY!

\section*{Declaration}

\vspace{10mm}

I, \Name{}, confirm that the work submitted in this dissertation is my own, and
that appropriate credit has been given where reference is made to the work of
others.
\\[8mm]
\noindent Signature:

% To add your signature, place the "signature.png" file in the "figures"
% folder and uncomment the line below (remove the leading %).

\begin{figure}[h]
  \includegraphics[width=50mm]{figures/signature.png}
\end{figure}

\noindent Date: \today

\newpage


  \section*{Acknowledgements}

Thank people, institutions, etc.


  % Add your abstract to the "abstract.tex" file. DO THIS PART LAST. It's
  % easier to write the abstract in the end (and do a good job of it).
  \section*{Abstract}

For a long time, models of the Moho discontinuity have been created from a variety of different methods including gravitational and seismological studies. However, very few of these models developed have uncertainty estimates, this is especially the case where there is limited seismic data available. In areas such as South America and Africa due to the economic and environmental challenges, over vast regions of the continent, there are little to no seismic point estimates. For these areas to have relevant Moho models either gravitational data needs to be used or the seismic data has to be interpolated; there is no way to tell how accurate these methods are to attaining a true Moho depth model. To determine a method's accuracy a method of cross-validation specifically repeated random sub-sample validation will be used to quantify errors on a gravitationally derived model of South America with the help of seismic point estimates. The results from this cross-validation will evaluate the accuracy of gravitational models in regions where there is no seismic data to compare it to. Additionally, for regions where the model significantly underestimates the Moho depth in comparison to the seismic data, there is likely an unmodelled mass present. The Paraná Basin, South America is thought to have large igneous intrusions resulting in a shallower Moho than expected. In this study, these intrusions will be modelled in an attempt to decrease the errors on the Moho model currently used. The cross-validation gives an indication that the specific model used has a reasonably small misfit from the point estimates but that the size of the error may vary geographically across South America as the error values attained are an average for the whole model.




  % Table of contents and lists of figures and tables are automatically
  % generated (yay!)
  \tableofcontents
  \listoffigures

  \pagestyle{fancy}

  \mainmatter

  % These are the actual chapters of your dissertation. Add your text, figures,
  % tables, etc to these .tex files.
  \chapter{Introduction}

The Mohorovičić discontinuity or Moho for short is the physical boundary signified by the change in many properties such as mineralogy, density, and temperature among other things, but it is mainly known as the change from the crust to the upper mantle. Ever since the discovery of this boundary observed through a significant change in seismic p-wave velocity on either side of the Moho by seismologist Andrija Mohorovičić in 1909 there have been many people using multiple methods to try and quantify the depths to this boundary. Some estimates of the Moho discontinuity include \cite{Laske2013}, \cite{Assumpo2013}, and \cite{Reguzzoni2015}, all of which have slightly different models depending on types of methods and data. Geophysical techniques such as seismology and gravimetry have been utilized to estimate the depth of the discontinuity and topography over a local to a global scale. This issue is known as a geophysical inverse problem and uses data in the form of gravity, seismic, or another method to determine the depth to the Moho over a certain area and produce a model.
Why is this important, the determination of the Moho is critical for many reasons. One of the main problems that it helps solve is knowledge about how plates deform and move, it is not the answer to the whole issue, but it is a contributing factor. Especially in regions of continental crusts where their behaviour is somewhat less predictable and understood when compared to oceanic plates. In continental settings, the crustal thickness is indicative of the stress of the lithosphere and helps with the determination through models of inter-plate faulting that often lead to earthquake events. The Moho thickness can also provide an insight into geothermal heat flow helping with heat flux models of the Earth where often without these depths to the crust-mantle interface these values need to be estimated as are unknown. Finally, determination of the thickness of the crust improves understanding of where deeper Earth minerals may be found. Or where specific minerals have been found in the past correlating this to a specific Moho thickness may indicate as to the type of setting to find explore in the future.
In this paper, the moho model will be derived from gravitational data and removing the local gravity disturbances to get the regional field that is almost entirely based on the depth to the Crust-Mantle boundary with the help of other parameters that need to be constrained including the density, reference Moho, and regularisation parameter. The problem with using gravitational data to model the Moho is that gravity values calculated to be the regional field do not only include the effect of the discontinuity but also the effects of unmodelled masses in the Earth's crust that have not been removed when taking away all other effects that contribute to the strength of the raw gravity data. These unmodelled masses lead to the emergence of uncertainties in the model created, however with these uncertainties being of an unknown magnitude then it is difficult to quantify and correct for these unknown masses as the location, size, density, and number are impossible to determine. This is not just the case in the \cite{Uieda2016}, but almost all gravitationally derived Moho models suffer the same fate including \cite{vanderMeijde2013}, \cite{Tugume2013}, and \cite{Reguzzoni2015}. All of these models produced are generally accurate representations of the Moho surface however none of these includes errors or uncertainty values to coincide with their models. Although these error estimates are often not seen in seismologically derived Moho models either, one of the only recognizable papers that try to calculate the uncertainty of their estimates is \cite{Szwillus2019} who obtain uncertainties by interpolating Moho depth even then these uncertainties calculated are larger than the errors coming from the P-wave velocity.
In this paper, the aim is to find uncertainties in a gravitationally calculated Moho model by using cross-validation with seismic constraints. With the hopes of finding the difference between a model calculated from gravity data and seismic point estimates to see how good the gravitational estimates of the depth to the moho are where there are not any seismic points that the model can constrain to. Continents such as Europe and North America have extensive seismological surveys that span most of the area so it would be considered pointless to see how well this method of uncertainty estimation works as there are no substantial areas of land without seismic data. This is why South America has been chosen as most of the surveyed areas by either reflection or refraction are situated along the coast with few based towards the centre of the continent mainly due to the difficultly of surveying a result of the magnitude and density of vegetation in the Amazon rainforest. However, South America is also limited with seismic data due to lack of financial funding as surveys over a large scale may not be economically viable for countries or companies that are interested in carrying out one. So with the vast area of the Amazon having little to no seismic points to constrain the gravitationally computed Moho discontinuity model it will be a good test to see how good the model is when there are no seismic point estimates to compare to.
Therefore this paper will use the Moho model of South America created by \cite{Uieda2016} and implement a cross-validation approach to calculate the uncertainties through the average differences between gravitational estimates and seismic point estimates of the crustal thickness or depth to Moho. The method builds upon the cross-validation originally used in the paper by randomly selecting a training and testing set and calculating not only the uncertainty but how many seismic point constraints are needed to accurately quantify the size of this error. In addition to using cross-validation, in the synthetic models created of South America, the effect of trying to model a previously unknown mass will be tackled by adding in underplating in the Parana basin addressed in \cite{Mariani2013}. Through this the possibility of locating unmodelled masses could be determined, if the cross-validation of seismic point estimates provides a difference from the gravity model then it is likely that in the area there is a denser/less dense mass in the crust depending on if the difference is positive or negative.

  \chapter{Data and Methodology}

To accurately obtain a model of the Moho depth one must first remove all other effects that contribute to overall gravitational values recorded over an area (see Figure~\ref{fig:gravity-correction}). This is achieved through removing the scalar gravity of an ellipsoidal reference Earth (the Normal Earth) and then the removal of all other effects. Initially, though the effect of the Normal Earth needs to be removed from the same point as to where the gravity observation was made and is calculated from the closed-form solution in \cite{Li2001a}. The value obtained here is called the gravity disturbance and can be seen in equation~\ref{eq:gravity_disturbance} below.
\begin{equation}
  \mathbf{\delta}(\mathbf{P}) =
    {g}(\mathbf{P}) -
    \mathbf{\gamma}(\mathbf{P})
  % Label used to reference the equation in the text.
  \label{eq:gravity_disturbance}
\end{equation}

\begin{figure}[h]
  \begin{center}
    % Width can be set to particular size (10cm) or relative to the page size,
    % like 0.5\textwidth (for half page) or \textwidth (for full page).
    \includegraphics[width=0.7\textwidth]{figures/gravity-correction}
  \end{center}
  \caption{
    Step by step stages of the removal of gravitational effects. (a) The measured gravity at point P$(g(P))$ with reference to the Earth. (b) The Normal Earth and normal gravity at P$(\gamma(P))$. (c) Removal of density anomalies e.g. oceans and topography to get $(\delta(P))$ the gravity disturbance. (d) The crust and mantle sources left after obtaining the Bouguer disturbance $(\delta_{bg}(P))$ by removing topography. (e) Assuming there are no unmodelled masses the remaining signal is the Moho and its corresponding depth. (f) Discretization of the anomalous Moho into tesseroids. Grey tesseroids have a negative density contrast while red ones have a positive contrast. After \cite{Uieda2016}.
  }
  % Label used to reference the figure in the text.
  \label{fig:gravity-correction}
\end{figure}
The gravity disturbance is still not a direct result of the change in density associated with the Moho discontinuity but also an amalgamation of topography with reference to the normal ellipsoid, variations of density in the crust (e.g. sedimentary basins and igneous intrusions), anomalies below the upper mantle, and mass deficiency due to the oceans. The effects of topography and sedimentary basins are removed through a topography correction in equation~\ref{eq:topography_correction}.
\begin{equation}
  \mathbf{\delta_{bg}}(\mathbf{P}) =
    \mathbf{\delta}(\mathbf{P}) -
    {g_{topo}}(\mathbf{P})
  % Label used to reference the equation in the text.
  \label{eq:topography_correction}
\end{equation}
This is the method used in \cite{Uieda2016} and assumes that the effects of other crustal and mantle sources are negligible, and after this correction, the gravitational values attained are purely a result of the density variations on either side of the Moho discontinuity. Seeing as the data is obtained for a sufficiently large area (South America) this correction for topography amongst other things is calculated using tesseroids as part of a spherical Earth approximation. Tesseroids (Figure~\ref{fig:tesseroids}) are spherical prisms that are used in place of rectangular prisms as they account for the curvature of the Earth. Effects of the tesseroids are calculated using a GLQ integration presented in \cite{Asgharzadeh2007} and improved upon in \cite{Uieda2015} through the adaptive discretization scheme.
\begin{figure}[h]
  \begin{center}
    % Width can be set to particular size (10cm) or relative to the page size,
    % like 0.5\textwidth (for half page) or \textwidth (for full page).
    \includegraphics[width=0.6\textwidth]{figures/tesseroid-coord-sys}
  \end{center}
  \caption{
   Sketch of a tesseroid system with geocentric coordinates (X Y Z). Gravitational observations are made at point P with respect to its local north-orientated coordinate system (x y z). From \cite{Uieda2015}.
  }
  % Label used to reference the figure in the text.
  \label{fig:tesseroids}
\end{figure}
\section{Overview of the methodology from Uieda (2017)}
As the cross-validation to estimate uncertainty in the model is implemented as part of the code used in the \cite{Uieda2016} paper the method in calculating the model is largely similar. For context, an overview of this method will be given but for more detail see \cite{Uieda2016}.
Upon the calculation of the Bouguer disturbance from the removal of topography, sediments etc. the forward model is parameterized by discretizing the anomalous Moho onto tesseroids. The forward model aims to calculate the difference between the Normal Earth Moho and the true Moho depth, and depending on which is shallower will result in either a positive (red) or negative (grey) density contrast displayed by the colour of the tesseroids, Figure~\ref{fig:gravity-correction}. The overall absolute value of the density contrast is a certain parameter, this produces a nonlinear problem with the equation~\ref{eq:forward_problem}.
\begin{equation}
  \mathbf{d} =
    f(\mathbf{p})
  % Label used to reference the equation in the text.
  \label{eq:forward_problem}
\end{equation}
Where $d$ is the data vector, $p$ is the parameter vector containing Moho depths, and $f$ is the non-linear function.
Leading on to the inverse problem the parameter vector is estimated using least-squares that reduces the misfit to the data, equation~\ref{eq:least_squares}.
\begin{equation}
  \mathbf{\phi}(\mathbf{p}) =
    {[\mathbf{d}^o - \mathbf{d}(\mathbf{p})]}^T
    [\mathbf{d}^o - \mathbf{d}(\mathbf{p})]
  % Label used to reference the equation in the text.
  \label{eq:least_squares}
\end{equation}
Where $d^o$ is the observed gravity data, the equation means that this is a non-linear inverse problem, but we can calculate the parameters using optimization, where a perturbation vector $\Delta p^0$ is iterated until a minimum is reached which leads to the value of $\phi(p)$.
The optimization of the least-squares estimate however is not enough for estimating the relief associated with the Moho and needs regularization in the form of a first-order Tikhonov regularization \cite{Tikhonov1977} to ensure smoothness in the model and avoid unstable solutions in Moho depth, to provide a realistic model \cite{Silva2001b}, see equation~\ref{eq:regularization} below.
\begin{equation}
  \mathbf{\theta}(\mathbf{p}) =
    \mathbf{p}^T\mathbf{R}^T\mathbf{R}\mathbf{p}
  % Label used to reference the equation in the text.
  \label{eq:regularization}
\end{equation}
$R$ a matrix composed of first-order differences between Moho depths. This along with the least-squares estimate leads to an inverse problem that is solved by minimising the goal function, equation~\ref{eq:goal_function},
\begin{equation}
  \mathbf{\Gamma(p)} =
    \mathbf{\phi(p)} + \mathbf{\mu}\mathbf{\theta(p)}
  % Label used to reference the equation in the text.
  \label{eq:goal_function}
\end{equation}
$\mu$ is the regularization parameter that helps control the fit to the observed data and the smoothness.
After the rearrangement and substitution of equations, we arrive at a linear equation system that can calculate the update ($\Delta p$) with reference to the Normal Earth Moho,
\begin{equation}
  [\mathbf{\mathbf{A}^k}^T\mathbf{A}^k + \mathbf{\mu}\mathbf{R}^T\mathbf{R}] \mathbf{\Delta}\mathbf{p}^k =
    \mathbf{\mathbf{A}^k}^T [\mathbf{d}^o - \mathbf{d}(\mathbf{p}^k)] - \mathbf{\mu}\mathbf{R}^T\mathbf{R}\mathbf{p}^k
  % Label used to reference the equation in the text.
  \label{eq:linear_equation_system}
\end{equation}
Where $Ak$ is the Jacobian matrix, and $\Delta p^k$ is the parameter perturbation vector.
Bott's method \cite{Bott1960} calculates the thickness of a sedimentary basin based on gravitational data, the method is iterative so recalculates a new vector of basement depths from the previous calculation until a value where the residuals (equation numerator) fall below the noise level, see equation~\ref{eq:bott_method}.
\begin{equation}
  \mathbf{\Delta}\mathbf{p}^k =
    \mathbf{d}^o - \mathbf{d}(\mathbf{p}^k) /
    {2}\mathbf{\pi}{G}\mathbf{\Delta}\mathbf{\rho}
  % Label used to reference the equation in the text.
  \label{eq:bott_method}
\end{equation}
Where $\Delta p$ is the density contrast between the sediment and the reference density, and $G$ is the gravitational constant (6.67x10-11 $m^{-3} kg^{-1} s^{-2}$). However, \cite{Silva2014} showed that Bott's method can be written as,
\begin{equation}
  \mathbf{A} =
    {2}\mathbf{\pi}{G}\mathbf{\Delta}\mathbf{\rho}\mathbf{I}
  % Label used to reference the equation in the text.
  \label{eq:special_bott}
\end{equation}
and the main advantage to this is that this method does not need the solution of the equation system, but rather a constant diagonal matrix, $A$. This scales the model depths to fit the gravitational data.
For calculating the depth to the Moho \cite{Uieda2016} uses Bott's method in the inversion process and adapts it onto a spherical coordinate system using tesseroids. Stating in this paper that this method retains the efficiency of Bott's method while accounting for the stability problem previously present in the method. Following this step, the final part of the method involves calculating the hyperparameters (regularization parameter $\mu$, Moho density-contrast $\Delta p$, and depth of the Normal Earth Moho $z_{ref}$) which will be used in the inversion process. The calculation of the regularization parameter is through a method of holdout cross-validation from \cite{Hansen1992} and from this optimal regularization value, the other two hyperparameters (Moho density contrast $\Delta p$, and depth of the Normal Earth Moho $z_{ref}$) can be calculated. The main way these parameters are calculated is by finding the smallest Mean Square Error (MSE) through a cross-validation method which compares known Moho depths from seismic point estimates to calculated ones from the hyperparameters and picks the values with the smallest associated MSE.
\section{Implementation of cross-validation in error estimation}
Cross-validation (CV) often used for large data sets in order to see how well the model produced from said data set performs independently (i.e. when there is no data to base the model on). The result of cross-validation is often an MSE or mean square error value which is the accuracy of the new predicted independent model. It is often the goal of the cross-validation in the first place, and how this can be minimised. For many cases of CV, the data set is split up into a training and testing set with the training set being used to find the best solution or model attained from the smallest cross-validation value while the testing (validating) set is kept separate. It is then compared to the model created from the training set to get an idea of the size of errors, and how well a model will perform for a completely independent set. The cross-validation procedure also helps prevent overfitting of the model or selection bias where some points tend to skew the overall model more than others. And in order to minimise these problems the most, the process is repeated multiple times with different training and testing sets along with the variation in the size of these subsets.
Many types of CV are relevant for different case-specific things, although mostly the methods are split into two main types: exhaustive and non-exhaustive. Exhaustive CV is where all possible combinations of separating the full data into training and testing sets are used leading to a limited number of iterations that can be run. This method often works best for small volumes of data as with larger sets the computational time becomes uneconomical and an overall waste of time. Non-exhaustive CV does not use all possible combinations but rather a large enough number of iterations to be considered representative of the full data set.
For this method in particular a procedure of non-exhaustive repeated random sub-sampling validation also known as the Monte Carlo method. It works by as the name states repeatedly selecting a random selection of the data into a training and testing set, demonstrated in Figure~\ref{fig:RRSSV},
\begin{figure}[h]
  \begin{center}
    % Width can be set to particular size (10cm) or relative to the page size,
    % like 0.5\textwidth (for half page) or \textwidth (for full page).
    \includegraphics[width=0.8\textwidth]{figures/RRSSV}
  \end{center}
  \caption{
   Procedure of repeated random sub-sample validation with the data split into different training (white) and testing (grey) sets for each iteration. After \cite{unknown}.
  }
  % Label used to reference the figure in the text.
  \label{fig:RRSSV}
\end{figure}
with the sizes of each set being determined by the user. The training data is used to find the best model or solution with the associated lowest cross-validation score, and this then being compared to the testing or validating set to find the associated errors on the model. The procedure used here is similar but not to be confused with the exhaustive counterpart leave-p-out cross-validation which is the exact same process except it uses all combinations of the data, which hasn't been used here as random sampling is easier to implement. The data used here is seismic point data that is compared to a gravitationally derived moho model from selected hyperparameters. All the different models from different hyperparameter combinations are weighed up against a training set of seismic point estimates to find the model with the smallest variance or best match to these point estimates. It is then compared to the rest of the seismic data "held back" to find the Mean Square Error (MSE) and subsequently the Root Mean Square Error (RMSE), which is the average uncertainty of the model in kilometres. With 100 iterations per size and there being 3 sizes of training sets each being the closest integer value to fractions 2/3, 3/4, and 4/5 of the full data this would lead to a large enough proportion of all possible combinations to attain a representative insight to how well the model performs for an independent set. The full data consists of 937 seismic point estimates from \cite{Assumpo2013}, of which the locations can be seen in Figure~\ref{fig:seismic_locations},
\begin{figure}[h]
  \begin{center}
    % Width can be set to particular size (10cm) or relative to the page size,
    % like 0.5\textwidth (for half page) or \textwidth (for full page).
    \includegraphics[width=0.5\textwidth]{figures/Seismic-locations}
  \end{center}
  \caption{
   Location and depth of seismic point estimates from \cite{Assumpo2013}. There are 937 total point estimates.
  }
  % Label used to reference the figure in the text.
  \label{fig:seismic_locations}
\end{figure}
meaning that the different training subset sizes are 625, 703 and 750, respectively. The training size always has to make up a larger proportion of the full data than the validating set as the model initially attained is representative of the overall data and hence the results of the MSE calculation is significant. This is supported by \cite{Berrar2019} who stated that 70-90\% of the full data should be part of the training set to be considered useful.
A single iteration of this procedure works through randomly selecting a select number of elements in an array of 937 points, these element positions are the data points in the latitude, longitude and seismic estimates selected to be part of the training set and the leftover elements not used are placed in separate latitude, longitude and seismic estimates array and are held back. The training set is used to find the best model through a function that gets the cross-validation scores for all solutions from which the best solution is selected that has the smallest CV score. This solution is then compared to the testing set arrays returning the MSE value between the best solution and the point estimates. This value is then stored in an array with all the other iterations and is then plotted as a histogram to find the mean and standard deviation of the MSE to obtain an estimate of the uncertainty of the model in predicting the Moho depth where there is no seismic data available.
However, with all methods comes the disadvantages, repeated random sub-sample validation (RRSSV) suffers from some randomly generated selection bias, where some datum may not be selected for any iteration as a part of the validating or testing subset but on the other hand some datum may have been selected multiple times, possibly skewing the MSE result. Additionally albeit unlikely testing sets selected for separate iterations may be identical, but this should not be a problem given the sufficiently large data set so the chance of exact same subsets in different iterations is very small.
\section{Using underplating to explain MSE values}
The RMSE values reached will likely not be negligible in comparison to the model and the reasoning behind this is unmodelled or hidden masses. For instance, when calculating the Moho depth for a point if an unmodelled mass is present and has a positive density contrast, in relation to the surrounding subsurface, then the gravity model will underestimate the depth of the crust-mantle boundary. To overcome this problem these hidden masses can be modelled and included in the model calculation to produce a gravitationally derived moho that has a depth more similar to that of the seismic point estimates for a region. \cite{Mariani2013} tried to overcome the unusually thick crust in the Paraná basin, Brazil when compared to simple isostatic models. This was done in the form of adding underplating in the area and seeing how it changes the Moho estimates in the area. Given the success of the method, the dimensions, and properties of the underplating will be implemented into this study by combining it with the full data. The summation of these two data sets helps calculate the Moho synthetic gravity anomaly. Although the exact values needed for the intrusion are not stated in the paper they can be estimated from a specific figure, see Figure~\ref{fig:underplating}.
\begin{figure}[h]
  \begin{center}
    % Width can be set to particular size (10cm) or relative to the page size,
    % like 0.5\textwidth (for half page) or \textwidth (for full page).
    \includegraphics[width=0.8\textwidth]{figures/underplating}
  \end{center}
  \caption{
   Underplating models with a $0.2Mg/m^3$ and $0.3Mg/m^3$ density contrasts for the Paraná Basin, South America. The models are calculated through inversion of gravity data. After \cite{Mariani2013}.
  }
  % Label used to reference the figure in the text.
  \label{fig:underplating}
\end{figure}
The values ultimately used here map out a square intrusion with a density contrast of 200kg/$m^3$ and a depth of -30km to -45km, the lateral extent of this underplating is from -55 to -49 degrees for the west and east longitude points respectively and -27 to -21 degrees for the north and south latitude dimensions. Adding this in should increase the Moho depth in the area, however, the most likely outcome is that it will increase the MSE averages reached in the cross-validation procedure. Although, it is an interesting avenue in calculating and adding previously unmodelled masses into models hopefully increasing the accuracy of gravitationally derived models.
\section{Software Implementation}
This inversion and error estimation method put forward in the methodology is executed in the Python programming language. Software is available under the BSD 3 clause open-source software license. The code in this project depend on open-source libraries scipy and numpy \citep{Harris2020} for number computations, matplotlib (\cite{Hunter2007}, \url{http://matplotlib.org}) and seaborn (\cite{Waskom2015}, \url{https://github.com/mwaskom/seaborn/tree/v0.6.0}) for plots and maps, Fatiando a Terra (\cite{Uieda2013a}, \url{http://www.fatiando.org}) for geophysics tasks. scipy.sparse package is implemented for use on sparse matrix arithmetic and linear algebra and solves the linear equation system equation.
The use of Jupyter notebooks (\cite{Perez2007}, \url{http://jupyter.org/}), which merge the source code, results, and figures of the project.
All source code, Jupyter notebooks, data, and error estimate results are available through an online repository (\url{https://github.com/compgeolab/moho-uncertainty}).
  \chapter{Results}

To test the cross-validation (CV) approach to error estimation this code has been added into the \cite{Uieda2016} synthetic-crust1 with Moho depth information extracted from the CRUST1.0 model \citep{Laske2013}. As previously mentioned this procedure needs the availability of seismic point estimates, the data for this is from Assumpção et al. (2013). These Moho depth estimates along with their geographical location can be seen in Figure~\ref{fig:seismic_locations} and in total there are 937 points.
The cross-validation approach used is repeated random sub-sample validation and as mentioned in the methodology randomly splits the full seismic data set into a training and testing (validating) set, with the training set compared to the solution to attain cross-validation values, the best solution is then selected from the smallest cross-validation value which is then scored against the testing set to attain the Mean Square Errors and subsequently the square root of these values which are the difference between the model and the point estimates. This error gives an indication to the average uncertainty in the overall model depth, and mainly to how good the model is where seismic data is not present which is largely the case for South America as most seismic point estimates are situated near the coast.
\section{Cross-validation results from the synthetic model}
In this run, the seismic point data will be split into a training and testing set, for a range of different proportions, these include the training size being 2/3, 3/4, and finally 4/5 of the full data. For each training size, the data that makes up this subset will be selected randomly from the full set for 100 iterations. The remaining data that was not selected for each iteration will be put into the validating set and held back for later scoring. As there are 937 separate seismic point estimates when the data is split these proportions need to be rounded to the nearest integer hence 2/3 is 625 data points, 3/4 is 703 points, and finally 4/5 rounds to 750 points. Figure~\ref{fig:histogram_no_intrusion}
\begin{figure}[h]
  \begin{center}
    % Width can be set to particular size (10cm) or relative to the page size,
    % like 0.5\textwidth (for half page) or \textwidth (for full page).
    \includegraphics[width=0.6\textwidth]{figures/no-intrusion-625}
    \includegraphics[width=0.6\textwidth]{figures/no-intrusion-703}
    \includegraphics[width=0.6\textwidth]{figures/no-intrusion-750}
  \end{center}
  \caption{
   RMS values from cross validation. Training sizes top to bottom are: 625, 703, and 750 all have 100 iterations. The dotted black line indicates the mean RMS value which is stated in the top right corner along with the standard deviation (std).
  }
  % Label used to reference the figure in the text.
  \label{fig:histogram_no_intrusion}
\end{figure}
shows the results of the cross-validation in the form of histograms showing the RMS values for all the iterations. All of these display somewhat of a normal distribution that should be more profound if more iterations were run. The mean values for all these histograms are very similar with all the values ranging between 2300-2350 metres with the highest value, 2344m, associated with the smallest training size of 625 and the smallest RMS value correlating to the largest training size. The standard deviation (std), which is a measure of the tightness of the spread to the mean value, increases with larger training sizes. This means that for larger training sizes the RMS values are more spread out with points for the largest training size of 750 having values that range from 1900-2700m with a standard deviation of 164.6. On the other hand, the other two sizes, 625 and 703, have std values of 103.0 and 124.8 respectively with RMS values not reaching below 2000m.
\cite{Szwillus2019} uses a similar method of cross-validation to estimate Moho uncertainty, except the method used is seismic interpolation and is on a global scale rather than just South America Figure~\ref{fig:uncertainty}.
\begin{figure}[h]
  \begin{center}
    % Width can be set to particular size (10cm) or relative to the page size,
    % like 0.5\textwidth (for half page) or \textwidth (for full page).
    \includegraphics[width=0.9\textwidth]{figures/Szwillus-uncertainty}
  \end{center}
  \caption{
   Interpolated Moho depth and the corresponding uncertainty values for South America. After \cite{Szwillus2019}.
  }
  % Label used to reference the figure in the text.
  \label{fig:uncertainty}
\end{figure}
The average Moho uncertainty calculated was 4.5km for South America however, the range of values was much larger with uncertainties in some places reaching 12km although these values were in places where no seismic data was present. This result is just over 2km higher than the mean uncertainty values seen in the histograms of around 2.3-2.4km and is likely due to the differing method.
However, when compared to the model, these RMS values are quite small in comparison to the Moho depths from the model which on average is probably between 30-40km across the continent, where most of the seismic point estimates are located. The difference between the individual point estimates and the model in that location is shown in Figure~\ref{fig:difference_no_intrusion}.
\begin{figure}[h]
  \begin{center}
    % Width can be set to particular size (10cm) or relative to the page size,
    % like 0.5\textwidth (for half page) or \textwidth (for full page).
    \includegraphics[width=0.9\textwidth]{figures/difference-no-intrusion}
  \end{center}
  \caption{
   Plot of estimated Moho depth without the added underplating with seismic point estimates superimposed. The colours of the point estimates (circles) represent the difference between the model and the seismic data.
  }
  % Label used to reference the figure in the text.
  \label{fig:difference_no_intrusion}
\end{figure}
The point estimates generally tend to agree with the model, however, in few places like the Andes the model is underpredicting the Moho depth when compared to the point estimates, this could have given rise to the higher standard deviation for the larger training sets as a majority of the points held back for the validating set for some iterations may have been points from the Andes. This is especially likely seeing as a reasonable proportion of the full 937 points are situated in the mountain range.
\section{Cross-validation results after adding in underplating}
This trial run is identical in every way to the synthetic-crust1 model except an intrusion has been added in the Paraná Basin. There are clusters of seismic point estimates situated in the same and surrounding area. Like the synthetic model the training sizes are 625, 703 and 750 with the full data set consisting of 937 points, each size was run with 100 iterations to create histograms of RMS values shown in Figure~\ref{fig:histogram_intrusion}.
\begin{figure}[h]
  \begin{center}
    % Width can be set to particular size (10cm) or relative to the page size,
    % like 0.5\textwidth (for half page) or \textwidth (for full page).
    \includegraphics[width=0.6\textwidth]{figures/intrusion-625}
    \includegraphics[width=0.6\textwidth]{figures/intrusion-703}
    \includegraphics[width=0.6\textwidth]{figures/intrusion-750}
  \end{center}
  \caption{
   RMS values from cross validation with Paraná Basin underplating included in the model. Training sizes top to bottom are: 625, 703, and 750 all have 100 iterations. The dotted black line indicates the mean RMS value which is stated in the top right corner along with the standard deviation (std).
  }
  % Label used to reference the figure in the text.
  \label{fig:histogram_intrusion}
\end{figure}
These histograms like those without the intrusion display a fairly normal distribution, however, the mean values are higher. This result was expected as the inclusion of the underplating will increase the difference in that area between the model and seismic point estimates. The mean values of each training size are 2552m, 2531m and 2491m respectively with the value decreasing as the training size increases, these are around 200m higher than the equivalent training size without the intrusion. Like the results of the model without the intrusion too the standard deviation increases with larger training sizes. In comparison, these standard deviations are higher with the std value for size 625 being 131.1 which is around 28m higher than its counterpart. The ranges of the RMS values though do not exceed 1900-2700m the higher std values are explained by a larger proportion of the values attained through cross-validation being near the edges of the range.
These results in comparison to the overall Moho depths are not that large as again the depths of the model are very similar to that of the model without the intrusion with the mean value being somewhere between 30-40km, so an error of about 2.5km or around 6-8\%. This error is low and so indicates that the model fits the seismic data very well. These discrepancies may in part be due to the Andes problem stated above but also to the large difference between the model and the point estimates in the Paraná Basin meaning that if the majority of these points with large disparities (see Figure~\ref{difference}) are selected as part of the testing set then the RMS value increases.
\begin{figure}[h]
  \begin{center}
    % Width can be set to particular size (10cm) or relative to the page size,
    % like 0.5\textwidth (for half page) or \textwidth (for full page).
    \includegraphics[width=0.9\textwidth]{figures/difference}
  \end{center}
  \caption{
   Plot of estimated Moho depth with the added underplating with seismic point estimates superimposed. The colours of the point estimates (circles) represent the difference between the model and the seismic data.
  }
  % Label used to reference the figure in the text.
  \label{fig:difference}
\end{figure}
\section{Software and run times}
For the code with and without the intrusion added both took 1hr and 57 minutes with 3 different training sizes and 100 iterations per individual size, i.e. 300 iterations in total. This was performed on a laptop computer with an AMD Ryzen 5 3500U 2.1GHz processor.
  \chapter{Summary}

Conclusion of your dissertation. Summarize what was learned and why it's
relevant.


  \pagestyle{plain}

  % Use the American Geophysical Union citation style
  \bibliographystyle{agu}
  % Use References instead of Bibliography (the default)
  \renewcommand{\bibname}{References}
  % The References section is automatically populated from the cited entries of
  % the references.bib file
  \bibliography{references}

  % If you want to have an appendix, uncomment these two lines and add an
  % appendix.tex file with the text. Sections declared in it will automatically
  % be numbered differently.

  %\appendix
  %\chapter{Appendix}

In order to create the plots and calculate results for this project we had to produce code through python language software in Jupyter lab. The repositories used were saved into the Github page Computer-Oriented Geoscience Lab ``compgeolab'' with the ones which provided all data presented named and described below:
\\ \footnotesize
\\
data/MarsTopo719.shape - Topography data file \\
data/gmm3\_120\_sha.tab - Gravity data file\\
environment.yml -directory that contains the collection of conda packages installed \\
prepare-gravity-grids.ipynb - Code required in order to export the now usable grids to a netCDF file \\
functions.py - Contains the utility functions for this project \\
mars-bouguer-density.ipynb - Code used to performed the calculations plotting figures for all regions used \\
Further-analysed-density.ipynb - Code used for plotting the figures whch were further analysed \\

\normalsize In order to test our code to see if the calculations would acquire an accurate density value we recreated the results of Carartori Tontini [2007]. 
\begin{figure}[H]
	\centering
	\subfloat{\includegraphics[width=65mm]{Figures/Caratori Tontini Recreation}\label{fig:C T recreation}}
	\subfloat{\includegraphics[width=71mm]{Figures/Density from Caratori Tontini}\label{fig:Density from C T}}
\end{figure}
The plots created obtained density value of $2300 kg/m^3$. This is slightly less than the $2400 kg/m^3$ from the paper but, this could be a consequence of using our gravity grid at a height of 10 km instead of 0 km. Despite that it does represent that to calculation performed in the code is successful in finding an accurate optimal Bouguer density; therefore, we can have confidence in the results shown in this Thesis.


\end{document}

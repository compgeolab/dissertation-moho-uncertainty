% Literature review template for the Geophysics programs of the University of
% Liverpool.
%
% This document sets the configuration and contains the main text of the
% review. References are managed by BibTex and kept in references.bib.

% Fill these in and they will be set throughout the document
\newcommand{\Name}{Yoür Name}  % Note that you can use any Unicode character
\newcommand{\Title}{Title of the literature review}

\documentclass[12pt,a4paper,onecolumn,oneside]{article}
% Full Unicode support for non-ASCII characters
\usepackage[utf8]{inputenc}
% Typographical rules for English
\usepackage[english]{babel}
% Handling figures in PNG, JPG, PDF, etc
\usepackage{graphicx}
% Better and more extensive maths
\usepackage{amsmath}
% Set the borders of the page
\usepackage[width=155mm,top=35mm,bottom=25mm,headsep=10mm,headheight=5mm]{geometry}
% Include links and metadata in PDFs
\usepackage[pdftex,colorlinks=true]{hyperref}
% Define command to insert month name and year as date
\usepackage{datetime}
% Nice styling for headers and footers
\usepackage{fancyhdr}
% Formatting the bibliography
\usepackage[round]{natbib}
% To insert dummy text into this template. Can be removed.
\usepackage{lipsum}

% Define a date format that is just the month and year
\newdateformat{monthyear}{\monthname[\THEMONTH], \THEYEAR}

% Setup metadata for the PDF and link colour
\hypersetup{
    pdftitle={\Title},
    pdfauthor={\Name},
    linkcolor=black,
    citecolor=black,
    filecolor=black,
    urlcolor=blue
}

% Set fancy headers
\usepackage{fancyhdr}
\pagestyle{fancy}
\fancyhf{}
\lhead{\fontsize{10pt}{0}\selectfont\itshape Literature Review -- \Name{}}
\chead{}
\rhead{\fontsize{9pt}{0}\selectfont \thepage}
\cfoot{}
\renewcommand{\headrulewidth}{0pt}

% Increase the line spacing
\renewcommand{\baselinestretch}{1.5}


\begin{document}


% Make the title section
\thispagestyle{empty}
\begin{center}
  {\LARGE \Title{}}
  \\[0.3cm]
  {\large \Name{}}
  \\[0.3cm]
  {\monthyear\today}
\end{center}


% Sections with an * aren't numbered
\section*{Abstract}

% Your abstract goes here.
% Remove all the "lipsum" lines below, they're just used to generate filler
% text.
\lipsum[1]



\section{Introduction}

The citations and bibliography are automatically handled by
BibTeX and the \texttt{natbib} package. Place the bibliographic information for
the references in the \texttt{references.bib} file and then cite your
references in the text (see below). The References section will be generated
automatically. To get the bibliographic information for papers, use the
\url{https://www.doi2bib.org/} website (DOIs are unique identifiers for
scientific publications, datasets, etc; you can find them in the paper PDFs or
publisher websites). \cite{Laske2013}

\subsection {seismology}

This is how you make citations: in the text \cite{Parker1973} or with
parenthesis \citep{Parker1973}. See
\url{https://www.overleaf.com/learn/latex/Natbib_citation_styles} for more
information.

See the files in the \texttt{dissertation} folder for examples of how to
included and cross-reference tables, figures, and equations.


\section{Some section}




\section{Conclusions}



% Use the American Geophysical Union citation style
\bibliographystyle{agu}
% Use References instead of Bibliography (the default)
\renewcommand{\bibname}{References}
% The References section is automatically populated from the cited entries of
% the references.bib file
\bibliography{references}

\end{document}
